\documentclass{article}

% Language setting
% Replace `english' with e.g. `spanish' to change the document language
\usepackage[english]{babel}

% Set page size and margins
% Replace `letterpaper' with `a4paper' for UK/EU standard size
\usepackage[a4paper,top=2cm,bottom=2cm,left=3cm,right=3cm,marginparwidth=1.75cm]{geometry}

% Useful packages
\usepackage{amsmath}
\DeclareMathOperator{\sinc}{sinc}
\DeclareMathOperator*{\argmax}{arg\,max}
\DeclareMathOperator*{\argmin}{arg\,min}

\usepackage{amssymb}
\usepackage{float}
\usepackage{graphicx}
\usepackage[colorlinks=true, allcolors=blue]{hyperref}

\usepackage{tikz}
\usetikzlibrary{arrows,decorations.pathmorphing,backgrounds,fit,positioning,shapes.symbols,chains,shapes.geometric,shapes.arrows,calc}

\usepackage{listings}
\usepackage{xcolor}
\usepackage{enumitem}
\usepackage{physics}
\setlist[itemize]{nosep}

\definecolor{codegreen}{rgb}{0,0.6,0}
\definecolor{codegray}{rgb}{0.5,0.5,0.5}
\definecolor{codepurple}{rgb}{0.58,0,0.82}
\definecolor{backcolour}{rgb}{0.95,0.95,0.92}

\lstdefinestyle{mystyle}{
  backgroundcolor=\color{backcolour},
  commentstyle=\color{codegreen},
  keywordstyle=\color{magenta},
  numberstyle=\tiny\color{codegray},
  stringstyle=\color{codepurple},
  basicstyle=\ttfamily\footnotesize,
  breakatwhitespace=false,
  breaklines=true,
  captionpos=b,
  keepspaces=true,
  numbers=left,
  numbersep=5pt,
  showspaces=false,
  showstringspaces=false,
  showtabs=false,
  tabsize=2
}

\lstset{style=mystyle}

\title{Computational Imaging}
\author{Matteo Galiazzo}

\begin{document}

\maketitle

\tableofcontents

\section{Introduction}

Image processing is the field of enhancing the images by tuning many parameters and features of the images.
Image processing is the subset of computer vision.
Here, transformations are applied to an input image and the resultant output image is returned.
In computer vision we extract insights from images and videos, input can be both image and video, and the output can be an interpretation, which is often non-visual.
In image processing input and output are both images, and the operations are at low-level and affect pixels within the image.

A continuous image is a function $f:\omega \in \mathbb{R}^2 \rightarrow R$.
A discrete image $A$ is a matrix of size $M \times N$ obtained by discretizing the function f.
The intersection between a row and a column is called pixel.
The value assigned to every pixel is the average brightness in the pixel rounded to the nearest integer value.
The process of representing the amplitude of the $2D$ signal at a given coordinate as an integer value with $L$ different gray levels is usually referred to as amplitude quantization.

Many techniques developed for the single channel image are repeated on the three channels.
Different manipulating operations can be performed on the images, such as:
\begin{itemize}
  \item Transforming the color space.
  \item Applying fine transformations.
  \item Create cartoonized images.
  \item Object detection using colors in HSV.
\end{itemize}

Computational algorithms are essential to convert data to images, since sensors almost never directly generate usable images.

The reconstruction algorithm necessarily requires the solution of an inverse problem.
We call two problems inverse one another if the formulation of the each involves part or all the formulation of the other.
The problem that is more extensively studied is usually called forward (or direct), the other is called inverse.
The direct problem is usually oriented along a cause-effect, in the sense that it computes the effect given the cause.
The inverse problem is associated with the reversal of the cause-effect and consists in finding the causes given the effects.
In direct problems usually there is a loss of information from the input to the output.
Hence, in the solution of the inverse problem it is impossible to recover the object exactly, due to the information lost in the direct one.

Linear inverse imaging problems are characterized by a linear model describing the relationship between the measured data $y$ and the unknown object $x$.
The direct linear model can be expressed as: $Ax = y$, where $A$ is a matrix representing the linear operator acting on the image $x$ to generate the data $y$.
The data $y$ are usually affected by noise, due to different physical processes.
For the moment, we can consider addittive noise so that the final linear model is: $Ax = y+e$.
This is the model we want to invert.
The noise is a random process described by a random variable.
Different random processes can corrupt the data $y$, represented by random variables with different probability distributions.



\end{document}
